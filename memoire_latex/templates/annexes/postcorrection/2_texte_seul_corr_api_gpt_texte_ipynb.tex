\begin{minted}[linenos,frame=lines,fontsize=\small]{python}
#!pip install openai

#from google.colab import drive
#drive.mount('/content/drive')

import os
from openai import OpenAI

client = OpenAI(
    api_key=[...],
)

def read_file(file_path):
    with open(file_path, 'r', encoding='utf-8') as file:
        return file.read()

def save_corrected_text(corrected_text, output_path):
    with open(output_path, 'w', encoding='utf-8') as file:
        file.write(corrected_text)

def process_files(input_directory, output_directory):
    os.makedirs(output_directory, exist_ok=True)

    for filename in os.listdir(input_directory):
        input_file_path = os.path.join(input_directory, filename)

        if os.path.isfile(input_file_path):
            file_content = read_file(input_file_path)

            chat_completion = client.chat.completions.create(
              messages=[
                  {
                      "role":"system",
                      "content": "Tu es un correcteur spécialisé dans la reprise de corpus textuels français bruités issus d'HTR ou d'OCR.",
                      "role": "user",
                      "content": "Voici un texte brut issu d'une transcription sommaire.\n"
                      "Corrige-le selon les règles suivantes et retourne un texte brut non balisé :\n"                      
                      "Règles de correction :\n"
                      "- Corrige l'orthographe, la syntaxe et la grammaire française.\n"
                      "- Respecte les crochets droits et les parenthèses.\n"
                      "- Place des majuscules en tête de phrase. Les phrases débutent uniquement après un point. Attention certains points sont en réalité des virgules.\n"
                      "- Place des minuscules à l'initiale des noms propres quand elles n'y sont pas.\n"
                      "- Place des minuscules à l'initiale des noms communs.\n"
                      "- Ne rajoute pas ou n'enlève pas de sauts à la ligne.\n"
                      "- Si tu rajoutes un mot, par exemple pour rendre 'voulez-vous' au lieu de 'voulez', place ce mot entre crochets droits. \n"                       
                      "- Rajoute des virgules et transforme des points et tirets en virgule seulement quand le contexte le permet. \n" 
                      #"- Supprime les espaces et les sauts à la ligne surnuméraires.\n"
                      #"- Si cela est pertinent, crées des listes en ajoutant des deux-points et des tirets. \n"
                      "Texte :" + file_content,

                  }
              ],
            model="gpt-3.5-turbo",
            temperature=0.7
          )

            response = chat_completion.choices[0].message.content


            output_file_path = os.path.join(output_directory, filename)
            save_corrected_text(response, output_file_path)

            print(f"Fichier corrigé enregistré : {output_file_path}")


input_directory = '/content/drive/MyDrive/Stage_IA/corr_test/input_text'
output_directory = '/content/drive/MyDrive/Stage_IA/corr_test/output_text'

process_files(input_directory, output_directory)
\end{minted}