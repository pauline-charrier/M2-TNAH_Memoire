\begin{minted}[linenos,frame=lines,fontsize=\small]{xquery}
(:~ 
    : Page chronologie | Timeline page
    :
    : @author Nom des auteurs, INHA https://www.inha.fr, ? License
:)

module namespace timeline = 'LettresDoucetReneJean/timeline';

import module namespace config = 'LettresDoucetReneJean/config' at 'config.xqm';
import module namespace app = 'LettresDoucetReneJean/app' at 'app.xqm';

(: Fonction rendant le squelette HTML de la page chronologie :)
declare
    %rest:path("LettresDoucetReneJean/timeline")
    %rest:GET
    %rest:query-param("mode", "{$mode}", "")
    %output:method('html')
    function timeline:fonction($mode) {
  
  let $csv := ''
  
  return
  
  <html lang="en">
    
    <head>
        <meta charset="UTF-8" />
        <meta name="viewport" content="width=device-width, initial-scale=1.0" />
        <title>Chronologie Jacques Doucet</title>
        
        <link rel="stylesheet" href="https://cdnjs.cloudflare.com/ajax/libs/font-awesome/5.15.3/css/all.min.css"/>
        <link href="https://cdnjs.cloudflare.com/ajax/libs/vis/4.21.0/vis.min.css" rel="stylesheet" />
        <link href="{$config:Serveur}/ressources/css/chrono.css" rel="stylesheet" />
        <link rel="stylesheet" href="https://cdn.jsdelivr.net/npm/bootstrap@4.3.1/dist/css/bootstrap.min.css"/>
        
        <script src="https://cdnjs.cloudflare.com/ajax/libs/vis/4.21.0/vis.min.js"></script>
        
        
        
        <script src="{$config:Serveur}/ressources/js/chrono.js"></script>
    </head>
    
    <body>
    
      <div id="serveur" style="display: none;">{$config:Serveur}</div> 

      <div id="visualisation"></div>
          <div id="controleurs">
               <button onclick="setRange('periode1')">Afficher la période 1850-1907</button>
               <button onclick="setRange('periode2')">Afficher la période 1908-1913</button>          
              <button onclick="setRange('periode3')">Afficher la période 1914-1929</button>
             
             
              
              <!-- <button onclick="setRange('decennie')">Affichage par décennie</button>
              <button onclick="setRange('annee')">Affichage par année</button>
              <button onclick="setRange('global')">Revenir à l&apos;affichage global</button>  -->             
              
              
              
          </div>
          <br/>
          <div class="container">
          <div id="telechargement">           
            
            
            <a href="{$config:Serveur}/csv/chrono3.xlsx" download="chrono3.xlsx">
              <button style="color:red; font-weight:bold;">Télécharger le fichier source au format Excel</button>
            </a> 
            
            <button style="display:none;" onclick="window.location.href='{$config:Serveur}/csv/chrono3.csv'">Télécharger le fichier source au format CSV</button>
            
             <a style="display:none;" href="{$config:Serveur}/csv/chrono3.txt" download="chrono3.txt">
              <button>Télécharger le fichier source au format texte</button>
            </a> 
            </div> 
            <br/>  
            
                   
           <div style="display:none;" class="row align-items-center">

            <div class="col-md-4 d-flex justify-content-center">
                <img class="img_info" src="{$config:Serveur}/img/pointing-right.png" alt="Icône indiquant une information venant à droite"/>
            </div>
            
            <br/>
            
            <div class="col-md-8">      
            <p class="instructions">
              <i class="fas fa-info-circle mr-3"></i>
              <u>Pour ouvrir le fichier CSV ou texte correctement dans Excel :</u>
              <br/>          
              <br/> 
              1. Télécharger le fichier (CSV ou texte) en cliquant sur le lien ci-dessus.
              <br/>
              2. Ouvrir un nouveau document Excel vierge.
              <br/> 
              3. Dans la barre supérieure, cliquer sur <strong>Données</strong>.
              <br/> 
              4. Puis cliquer sur la deuxième icône en partant de la gauche <strong>A partir d&apos;un fichier texte/CSV.</strong>
              <br/> 
              5. Dans l&apos;assistant d&apos;importation qui apparaît, choisissez les options suivantes :
              <br/>     
              <i class="fas fa-arrow-right"></i> dans le menu déroulant <strong>Origine du fichier</strong>, sélectionner <strong>65001:Unicode (UTF-8)</strong>
              <br/> 
              <i class="fas fa-arrow-right"></i> dans le menu déroulant <strong>Délimiteur</strong>, sélectionner <strong>Tabulation</strong>
              
          </p>
          
          </div>

           </div>

          </div>
          
    </body>

</html>
    };

(: Fonction (JCC) pour lire les fichiers de formats variés :)
declare
  %rest:path('/LettresDoucetReneJean/{$file=.+\.(csv|jpg|geojson|txt|xlsx)(\?.+?=.+?)?}') (:ajouter le geojson et le XLSX :)
  %output:method('basex')
  %perm:allow('public')
function timeline:fichierCSV(
  $file  as xs:string
) as item()+ {
  let $path := file:base-dir() ||  $file
  return (
    web:response-header(
      map { 'media-type': web:content-type($path) },
      map { 'Cache-Control': 'max-age=0,public', 'Content-Length': file:size($path) }
    ),
    file:read-binary($path)
  )
};  

(: Fonction (PC) pour extraire de la base les informations utiles pour l'intégration des lettres dans la chronologie :)

declare
    %rest:path("/LettresDoucetReneJean/timeline/documents")
    %rest:GET
    %output:method("text")
    function timeline:documents() {
        let $bdd := $config:BDD
        let $docs := for $doc in collection($bdd)
            let $when := $doc//*:profileDesc/*:creation/*:date/@when
            let $from := $doc//*:profileDesc/*:creation/*:date/@from[not(starts-with(., '2024'))] (: Exclure l'anomalie :)
            let $id := $doc//*:sourceDesc//*:idno/text()
            let $titre := $doc//*:title[@type='main']/text()
            where $when or $from
            return
                concat(
                    $id, ';', 
                    if ($when) then $when else $from, ';', 'lettres', ';', $titre
                )
        let $csvHeader := 'id;date;type_event;titre'
        let $csvContent := string-join($docs, '&#10;') (: Retours à la ligne :)
        return concat($csvHeader, '&#10;', $csvContent)
    };
\end{minted}