Le concept d’édition numérique, compris dans son acception la plus large, à savoir celle d’une publication textuelle produite et accessible sous format électronique, à l’histoire pas si récente et fortement liée à celle du web, recouvre de nos jours des réalités très vastes, allant de la mise à disposition de documents analogiques numérisés, ou de livres adaptés à la lecture sur écran (désignés comme « e-books ») à la production de documents nativement numériques, issus d’une chaîne éditoriale spécifique, distincte de celle employée dans le monde de l’édition papier. 

Les initiatives ayant mené à la mise à disposition à grande échelle de versions électroniques de documents papier sont nombreuses et généralement enracinées dans le principe de démocratisation de l’accès à la connaissance, cher aux inventeurs du Web. Il est ainsi peu surprenant de constater qu’en raison de cette aspiration commune comme des fonctionnalités technologiques permise par la toile (circulation documentaire massive et rapide), cette dernière apparaisse comme un acteur majeur dans la diffusion et la structuration de cette pratique. 
Le Web, système de distribution d’informations appuyé sur Internet, inventé en 1989 par Tim Berners-Lee et rendu accessible au public dès 1991, porte ainsi comme l’un de ses principes fondateurs la facilitation de l’accès au savoir et la libre circulation de l’information\footnote{« The web should empower an equitable, informed, and interconnected society. It has been, and should continue to be, designed to enable communication and knowledge-sharing for everyone. »} \footcite{noauthor_ethical_nodate}. C’est l’adhésion (entre autres\footnote{Un autre enjeu fondamental résidant dans la préservation des œuvres menacées ou peu connues par exemple.}  )  à ce même principe qui, a pu prévaloir dans l’effort progressif de numérisation massive\footnote{ Réalisé soit par action mécanique (le texte a été re-tapé) soit par procédé photographique (numérisation en mode image).} (bientôt amplifié par les technologies et la visibilité offertes par le web) dès les années 1990. Citons à ce propos le Projet Gutenberg (né en 1971) et les efforts mené par les grandes bibliothèques (\bnf et Library of Congress dès 1990), les grandes entreprises (Google, 2004) ou des initiatives privées (Internet Archive, 1996)  vers la constitution de bibliothèques virtuelles. 

Le type d’édition numérique sur lequel nous nous concentrerons dans ce mémoire est cependant bien distinct du type d’approche évoqué plus haut. Il s’agit de l’édition numérique scientifique de sources historiques ou littéraires. Héritière du champ de l’ecdotique\footcite{gvelesiani_quest-ce_2017} et de la critique textuelle traditionnellement rattachée (au moins dans la tradition française) à l’étude philologique, l’édition scientifique, ou savante, revêt néanmoins elle aussi des formes diverses (documentaire, diplomatique, génétique…), d’autant plus variées à l’ère du numérique du fait de la multiplication des fonctionnalités et des facultés offertes au lecteur, qui permettent autant d’angles d’approches du texte. L’édition critique de sources primaires, composante majeure du travail de l’historien\footcite{poupeau_ledition_2008}, constitue, avec peu de surprise, l’un des principaux champs d’études ainsi qu’une dimension non négligeable des projets de recherche en humanités numériques\footcite[p.8]{earhart_digital_2012}   \footcite[paragr. 8]{hayles_humanites_2016}   \footcite{anderson_teaching_nodate}.

Les humanités numériques, largement définies comme un ensemble de « pratiques de recherche » en sciences humaines et sociales intégrant l’application d’outils et de méthodes computationnelles\footcite[p.33]{ manovich_data_2015 } au sein de leur processus\footcite[p.90]{ bermes_patrimoine_2020}, (au même titre – idéalement – qu’une réflexion sur ces mêmes outils et méthodes\footcite{anderson_teaching_nodate}), sont issues d’une histoire faite d’explorations et d’expérimentations, que de nombreux commentateurs, en quête d’historicisation du concept\footcite[p.33-49]{clavert_humanites_2019} ou de légitimité scientifique, comme le remarque Aurélien Berra\footcite[p.613-629]{ berra_pour_2015}, font remonter jusqu’au milieu du XXème siècle en la figure pionnière et quasi-« mythique » \footcite[p.315-318]{gefen_humanites_2017} du jésuite Roberto Busa. Ce dernier est l’auteur d’une lemmatisation complète (\textit{L’Index Thomisticus}), assistée par méthodes computationnelles, des œuvres de Thomas d’Aquin, débutée en collaboration avec l’entreprise IBM dès 1949 et dont le travail monumental ne sera publiée qu’en 1980\footcite{ mounier_ibm_2018}. Notons qu’une autre figure pionnière, bien moins souvent citée, est celle de Josephine Miles\footcite{bermes_patrimoine_2020}, dont la publication du travail de concordance, qui présente de nombreuses similarités avec celui de Busa, a précédé de dix-sept ans celle du premier volume de l’\textit{Index} de Busa\footcite{heffernan_search_2018}.  L’approche computationnelle qui fait l’identité des humanités numériques\footnote{Et à laquelle on peut également rattacher comme précurseur le mouvement de l’histoire quantitative et indirectement de l’histoire sérielle, portés notamment par l’Ecole des Annales dès les années 1950. Notons également l’approche « cliométrique » appuyée sur des méthodes statistiques.}   n’est donc pas nouvelle (elle apparaît en effet bien aussi ancienne que l’informatique elle-même), bien que les technologies employées et les potentialités envisagées aient fait l’objet d’une évolution remarquable au cours des dernières décennies, du fait de l’accroissement des puissances de calcul des ordinateurs et de l’adoption universelle du Web. La terminologie utilisée pour identifier le mouvement a elle-même également évolué, du restrictif  \textit{linguistic computing} des années 1960 (reflétant la part majoritaire des travaux en traitement automatique du langage naturel) au \textit{humanities computing} des années 1980-1990\footcite{burnard_quest-ce_2015} à la notion maintenant en passe d’être majoritairement adoptée\footcite[p.5]{bonfait_humanites_2021} de « \textit{digital humanities »}  \footcite{bermes_patrimoine_2020}, ayant progressivement émergé dans les années 2000\footnote{ Avec notamment la publication du manuel \textit{A Companion to Digital Humanities}  (Susan Schreibman, Ray Siemens, and John Unsworth) en 2004. }   avant d’être finalement consacrée par la publication du \textit{Manifeste des Digital Humanities}\footcite{dacos_manifeste_2011} en 2010. Ce dernier définit les humanités numériques comme véritable champ interdisciplinaire, une « transdiscipline », en mettant l’accent sur les pratiques de recherche ainsi que sur le rôle heuristique joué par le traitement informatique. Cette définition se détache donc en partie d’une précédente approche plus prudente, plus englobante et volontairement imprécise quant à la circonscription du territoire scientifique des humanités numériques, conçues comme un « grand chapiteau » (\textit{big tent}) et mettant en avant une certaine fluidité.
L’étude textuelle constitue, nous l’avons vu avec les exemples canoniques de Busa et de Miles, le champ d’application premier des humanités numériques (avec ce qui sera appelé le \textit{literacy and linguistic computing}\footcite{burnard_du_2012}), et il en demeure encore l’un des plus vigoureux. L’édition numérique savante de sources historiques s’inscrit pleinement dans ce mouvement, en ce qu’elle exploite les fonctionnalités offertes par le numérique, non seulement en ce qui concerne les fonctionnalités offertes par l’interface graphique d’un ordinateur (navigation facilitée par l’usage de l’hypertexte\footcite[p.17]{duval_pour_2017}, visualisations alignées entre édition diplomatique ou fac-similé et apparat critique par exemple) mais aussi par les outils développés spécifiquement pour un usage en humanités numériques (encodage, analyse statistique, indexation permettant de « multiples entrées » dans le texte\footcite[p.82]{chateau-dutier_editions_2021}), conférant ainsi à l’édition numérique une plus-value par rapport à son homologue sous format analogique\footnote{Certaines éditions numériques ont ainsi pu être jugées « meilleures » que leur version papier, comme l’édition de la correspondance de Van Gogh, ainsi citée par \footcite[p.80]{chateau-dutier_editions_2021}}, en ce qu’elle offre au lecteur une agentivité ouverte dans sa relation au texte par sa modularité\footcite[p.21]{duval_pour_2017}, multipliant les niveaux d’analyse et d’interprétation\footcite{casenave_reception_2017} de la source sans pour autant « altérer son intégrité documentaire » \footcite[p.82]{chateau-dutier_editions_2021}. 
Au sein des humanités numériques, les éditions électroniques de textes historiques et/ou littéraires constituent un champ d’application non négligeable et ancien. Le \textit{Catalogue of Digital Editions}\footcite{ucl_centre_for_digital_humanities_digital_nodate} (non exhaustif), issu d’une collaboration entre partenaires universitaires autrichiens et britanniques recense 337 projets d’éditions numériques en juillet 2024. Les projets pionniers ont vu le jour dès les balbutiements du web, citons parmi eux la \textit{William Blake Archive} par Morris Eaves, Robert N. Essick et Joseph Viscomi, la \textit{Dante Gabriel Rossetti Archive}\footcite{mounier_ce_2018}  \footcite{chateau-dutier_editions_2021} par Jerome McGann, ainsi que l’édition numériques des \textit{Contes de Canterbury} de Chaucer\footcite{casenave_reception_2017} par Peter Robinson, projets tous trois entrepris dès 1993.
Nous l’avons mentionné plus tôt, le domaine de l’édition scientifique, qu’elle soit numérique ou « analogique », comporte différents modèles d’application, correspondant à différents objectifs épistémologiques, différentes méthodes d’analyse et différentes traditions savantes, parmi lesquelles :
\begin{itemize}
\item l’édition diplomatique qui vise à une restitution la plus fidèle possible du texte (apparentée à une édition imitative) sans interventions ou le moins possibles de la part de l’éditeur, en conservant les passages moins intelligibles et la mise en forme d’origine\footcite[p.178]{masai_principes_1950}. Elle peut être accompagnée d’un apparat critique explicitant ou contextualisant le texte.

\item l’édition critique qui vise à établir une version du texte la plus fiable et cohérente possible (la « meilleure version du texte »), après avoir effectué des corrections éventuelles et confronté les différents états du texte (la collation), tout en proposant une documentation précise sur les choix éditoriaux réalisés et les variantes observées dans l’apparat critique. Ce type d’édition peut, dans une approche philologique, s’intéresser à la généalogie de la tradition du texte, notamment dans le cas d’une multiplicité des témoins, exercice appelé \textit{stemma codicum} selon la méthode de Lachmann\footcite[p.125-138]{fornaro_karl_2011}.

\item l’édition génétique qui se focalise sur le processus d’élaboration du texte, en mettant en évidence les différentes étapes de sa rédaction (les différents « états »), également accompagnée d’un apparat critique étudiant ces évolutions.

\item l’édition fac-similé, apparentée à l’édition dite documentaire, qui est une représentation de type photographique de la source\footcite[p.178]{masai_principes_1950}, distincte de l’édition diplomatique en ce qu’elle restitue une dimension matérielle généralement absente des autres modèles d’éditions.

\item l’édition variorum (ou « des variantes ») \footcite[p.82]{chateau-dutier_editions_2021}, présente le texte adossé à une documentation complète et approfondie sur les variantes textuelles (en ne limitant donc pas à une seule version définitive) et les interprétations et commentaires critiques des éditeurs précédents. Il s’agit d’abord de présenter une ressource exhaustive pour les chercheurs.
\end{itemize}
Définie par l’historien allemand Patrick Sahle comme « la représentation critique d’un document historique » \footcite{sahle_what_2016}, selon une définition volontairement large comme le souligne Frédéric Duval\footcite[p.15]{duval_pour_2017}, l’édition savante fait donc l’objet d’une « spécialisation » (généralement orientée vers l’approche génétique ou philologique) en fonction des objectifs scientifiques.

Le projet PENSE (pour Plateforme d’édition numérique de sources enrichies) porté par l’Institut national d’histoire de l’art (INHA) s’inscrit lui dans une approche mixte entre l’édition diplomatique et l’édition critique. Visant à terme le développement d’un portail unique de mutualisation (selon une logique déjà entamée avec la plateforme AGORHA\footnote{AGORHA (pour « Accès Global et Organisé aux Ressources en Histoire de l’Art ») est la plateforme de mutualisation des bases de données produites et maintenues par l’\inha et adossées à des programmes de recherche menés par l’Institut. Ayant fait l’objet d’une refondation intégrale en 2021, elle est accessible à l’adresse suivante : https://agorha.inha.fr/}  ) des différentes éditions numériques produites dans le cadre de programmes de recherche portés par l’\inha, il est conçu pour l’instant comme une sorte de « super-projet » englobant de multiples projets d’édition numérique aux objets divers. Lancé en 2020 par le jeune Service numérique de la recherche de l’\inha (fondé en 2019 et héritier de l’ancienne Cellule d’ingénierie documentaire (CID) \footcite{inha_service_nodate}), PENSE porte un double objectif de valorisation patrimoniale et d’exploitation scientifique, reflété par la collaboration des deux départements de l’Institut pour son déploiement (le département des études et de la recherche – DER, et le département de la bibliothèque et de la documentation – DBD). Double objectif également révélé par l’ambition de pouvoir permettre un accès aux sources selon « trois niveaux de lecture » \footcite{inha_produire_nodate} : un niveau « fondamental », constitué par la mise à disposition des données encodées et enrichies, un niveau « de médiation », constitué par la présentation de parcours thématiques facilitant l’entrée dans l’édition pour le public non spécialiste, et un niveau « critique », à visée plus technique, composé d’articles scientifiques.
Compte tenu de l’activité éditoriale traditionnelle de l’INHA\footcite{inha_editions_nodate} et d’un certain attachement à la bibliophilie liée tant à son héritage institutionnel (la Bibliothèque d’Art et d’Archéologie de Jacques Doucet) qu’à la tendance notée par Emmanuel Château-Dutier à la préférence des acteurs de l’histoire de l’art pour l’édition papier\footcite[p.86]{chateau-dutier_editions_2021}, il peut être intéressant de s’interroger sur cette orientation vers l’édition numérique au sein d’une telle institution. 
En histoire de l’art comme dans d’autres disciplines, les éditions numériques pionnières sur le web sont fortement marquées par une proximité formelle avec l’objet livre, jusqu’à chercher à « restituer » une matérialité de support avec la publication sous forme de CD-Rom\footcite{earhart_digital_2012}. En dépit de cette proximité formelle des débuts, il est difficile de ne pas constater un changement des modes de consultation de la source, comme le souligne Frédéric Duval (avec une transition de la figure du « lecteur » à celui de « l’utilisateur ») \footcite[p.20]{duval_pour_2017}. Ces changements, qui recoupent les questions de rapport à l’interface de visualisation et aux nouvelles fonctionnalités permises par le recours aux technologies numériques, donnant ainsi un nouveau rôle au « lecteur » sont également à mettre en relation avec les évolutions des enjeux d’accessibilité des éditions critiques, dont les publics visés ne sont pas nécessairement les mêmes selon qu’il s’agit d’une édition numérique, généralement librement accessible ou d’une édition papier, souvent coûteuse et peu intelligible pour le public non spécialiste\footcite{duval_pour_2017}  \footcite{gvelesiani_quest-ce_2017}.
Il faut également noter que l’histoire de l’art présente une tradition ancienne et riche d’édition savante et critique, aspect fondamental pour la constitution du champ comme « discipline historique » \footcite{chateau-dutier_editions_2021}. Une tradition qui s’est rapidement tournée vers l’informatique et la mise en ligne, avec des éditions pionnières remarquables comme celles de Vasari par Paola Barocchi (1997, mise en ligne en 1999) et de la correspondance de Van Gogh (débuté en 1994) par le Van Gogh Museum et le Huygens Institute\footcite[p.78-80]{chateau-dutier_editions_2021}.  
PENSE, qui s’intéresse tant à l’édition de corpus textuels qu’iconographiques, revendique son intégration à une \textit{digital art history}\footcite{inha_produire_nodate}, tel que définie par Johanna Drucker, c’est-à-dire une approche fondée sur le recours à des techniques d’analyse permises par la technologie computationnelle\footcite[p.7]{drucker_is_2013}, distinguée en cela clairement de la \textit{digitized art history}, désignant le recours aux entrepôts de données relatives à des objets préalablement numérisés. Cette distinction rappelle l’opposition dessinée entre \textit{digitizing art history} et \textit{computational art history} par Diane Zorich en 2012\footcite[p.5]{bonfait_humanites_2021}. Les deux termes sont cependant souvent utilisés de manière interchangeable comme le soulignent certains auteurs\footcite[p.59]{aubry_artificial_2021}.
La \textit{digital art history} selon Drucker doit également s’intéresser à la redéfinition induite par la médiation de la digitalisation (numérisation ou traitement par des méthodes et outils computationnels) de l’objet étudié, texte ou image (la numérisation étant déjà par elle-même un processus interprétatif et donnant lieu à des biais\footcite[p.12]{drucker_is_2013}), entraînant la nécessité d’une réflexion sur les usages de ces outils et leur impact sur la manière de construire l’histoire de l’art\footnote{Comme le soulignent Bonfait et al., reprenant la formule de Johanna Drucker, les « données ne sont pas données » (« data » contre « capta »).}.

Cette dimension réflexive de la \textit{digital art history}, revendiquée par le mouvement des humanités numériques, est aussi présente au sein de PENSE. Peut-être d’autant plus que le projet PENSE ceci de particulier que, du fait de son appartenance à une institution telle que l’\inha, doublement patronnée par les mondes de la recherche et de la culture, il n’ambitionne pas seulement de produire des éditions à caractère savant, mais également des éditions grand public des sources. C’est-à-dire des éditions capables de susciter chez le public non spécialiste, un intérêt intellectuel et une émotion esthétique, tout en conservant une dimension de rigueur scientifique exploitable par un chercheur. Ce caractère double, qui se distingue de bon nombre de projets « traditionnels » portés par les institutions de recherche ou muséales, amène à porter un regard rapidement critique et ouvert sur les décisions à mettre en œuvre.
C’est bien ce double aspect qui nous intéresse ici, et c’est cette double casquette, intrinsèquement liée à l’histoire et au rôle de l’établissement qui l’accueille et le porte, qui, à mon sens, permet d’expliquer à la fois les choix technologiques, « managériaux » ainsi que les angles d’approche du projet PENSE, parfois inhabituels, et qui font son originalité.



Dans ce mémoire nous étudierons la manière dont cette conceptualisation de la pratique de recherche en humanités numériques appliquées à l’histoire de l’art imprègne la conception et la mise en œuvre de la chaîne de traitement de données d’édition numérique au sein du projet PENSE. 

Après avoir présenté les acteurs à l'œuvre et les objectifs envisagés par le projet PENSE, nous nous intéresserons à la mise en place concrète de la chaîne de traitement des sources,  en étudiant tant le processus d'acquisition, de gestion et d'enrichissement des données textuelles, que la phase de valorisation, passant par la visualisation et la conception d'interfaces utilisateur.
Puis, nous étudierons la philosophie adoptée par le projet, entre un regard volontiers critique posé sur l'usage de la technologie dans la recherche en humanités et un intérêt fort porté sur des méthodologies issues du monde de l'industrie, jugées pertinentes dans le contexte du développement de projets numériques en collaboration avec des acteurs non spécialistes, comme le sont une partie des chercheurs en humanités « classiques ». 
Enfin, nous nous pencherons sur la dimension expérimentale mise en avant par le projet, en prenant pour cas d'étude l'usage de l'intelligence artificielle neuronale pour une tâche spécifique de post-correction de corpus textuels, intervenant en fin de chaîne de traitement dans le cadre de l'édition numérique de la correspondance de Jacques Doucet (grande figure de l'héritage de l'\inha) avec son bibliothécaire René-Jean.
