Le projet de plateforme d’édition numérique de sources enrichies développé par l’Institut national d’histoire de l’art (\inha) est particulièrement intéressant à étudier, en ce qu’il mêle, par une approche originale, des pratiques bien établies dans le monde des humanités numériques à des méthodes et aspects venus de mondes a priori bien éloignés, tels que ceux du commerce, de l’industrie et du service. En effet, si le projet intègre pleinement dans sa chaîne de traitement des formats et des technologies déjà couramment adoptés dans le paysage français des humanités numériques, comme le format d’encodage \tei et la visualisation interactive de données, il incorpore également des approches moins fréquemment croisées, comme le \textit{design thinking}, ancré dans le monde industriel, et les méthodes Agile, qui, si elles ne sont pas entièrement étrangères au monde de la recherche (notamment anglo-saxonne, où leur importation a pu donner lieu à des controverses critiques), sont encore bien loin d’être des standards dans les humanités numériques françaises, comme le souligne l’ingénieur en charge du développement du projet, Jean-Christophe Carius. Leur introduction dans un projet d’édition numérique en histoire de l’art, semble donc a priori suggérer une volonté, ou du moins une disposition de l'\inha à expérimenter avec des outils issus d’autres secteurs. Dans une institution comme l’\inha, que l’on imagine volontiers conservatrice du fait de son héritage (marqué par le don de la Bibliothèque d’Art et d’Archéologie de Doucet, grand bibliophile), l’existence de \pense démontre qu’il existe une certaine appétence pour l’innovation, bien qu’encore prudente et timide. Notons que le stage qui a servi d’appui à la rédaction de ce mémoire, a été le lieu de premiers essais en ce qui concerne l’introduction et l’exploitation de l’intelligence artificielle pour le traitement de la langue dans le cadre de la valorisation scientifique et patrimoniale des sources de l’histoire de l’art. 

L’une des dimensions centrales du projet de plateforme \pense réside dans sa philosophie, héritée notamment des principes du Web, et axée sur l’ouverture, la transparence, l’ergonomie (tant pour le chercheur que pour l’utilisateur final) et l’accessibilité. \pense propose ainsi, en adéquation avec la politique impulsée par le \snr et en écho avec la démarche vers la science ouverte encouragée par le Ministère de l’Enseignement et de la Recherche, une démarche d’ouverture des données et des sources, en privilégiant une diffusion large et libre des connaissances produites. Cette démarche est voisine d’une approche privilégiant l’intelligibilité des technologies utilisées. 

Car si \pense intègre certains concepts issus de l'industrie, d'autres approches qui pourtant partagent ce même héritage, notamment celles orientées vers la performance technique et la complexité des développements applicatifs, sont délibérément écartées. Cela reflète la philosophie affirmée par les acteurs du projet, pour lesquels la technologie doit être envisagée avant tout comme un outil au service de la recherche, et non comme une fin en soi. Ce regard critique sur la place de la technologie se distingue d’autres initiatives en humanités numériques, plutôt axés sur l’ingénierie, où les figures de l’ingénieur et du chercheur s’éloignent, ou au contraire se confondent, voire sont fusionnées. Au contraire, dans le projet \pense l’approche choisie est plutôt de mettre en avant la complémentarité de ces deux rôles et d’affirmer l’importance capitale du dialogue entre les deux. Dialogue appuyé et encouragé notamment précisément par la mise en place de méthodologies comme celle du \textit{design thinking}, centré sur une démarche itérative impliquant le développeur comme « l’utilisateur » du début à la fin du projet.

Les limites technologiques rencontrées par le projet \pense sont relativement typiques de celles que l’on observe fréquemment dans le domaine des humanités numériques. Parmi celles-ci figurent le choix, en dépit des possibilités technologiques, de privilégier les tâches manuelles pour la correction des textes, en raison des performances encore limitées des technologies de reconnaissance optique de caractères (OCR) et de reconnaissance automatique de l’écriture manuscrite (HTR), qui ne leur permettent pas encore de satisfaire en autonomie les exigences de qualité des données dans la recherche. 

Enfin, il s’agit de rappeler encore une fois le double rôle scientifique et patrimonial porté par le projet \pense et incarné dans la « co-départementalité » à l’œuvre dans sa mise en place (Département de la Bibliothèque et de la Documentation et Département des Etudes et de la Recherche). Ce double rôle, observable dans un double objectif, est particulièrement perceptible, tant du point de vue du choix des technologies choisies (qui doivent rester relativement accessibles pour les chercheurs et ergonomiques pour les utilisateurs hors des murs de l’\inha), que du développement des interfaces (double public) ou encore du traitement appliqué aux sources. Cette double dimension est reflétée par le double patronage de l’institution hôte, placée sous l’égide du Ministère de la Culture et du Ministère de l’Enseignement supérieur et de la Recherche.  
