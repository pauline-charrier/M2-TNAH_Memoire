\chapter{Résumé}

\textbf{Résumé :}
	Le présent mémoire a été réalisé dans le cadre de la deuxième année du master « Technologies numériques appliquées à l’histoire » de l’Ecole nationale des chartes. Il s’inscrit à la suite du stage de fin d’études réalisé au Service numérique de la recherche de l’Institut national d’histoire de l’art et ayant eu lieu d’avril à juillet 2024. Le stage s'est concentré sur l'étude de la chaîne de traitement appliquée par le projet PENSE (2020-) pour la publication d'éditions numériques, du traitement de corpus textuels (XML-TEI) à l'exposition des données sur le web (développement d'interfaces utilisateur) en passant par la visualisation de données (Leaflet, Vis) et la correction automatique assistée par intelligence artificielle neuronale. Une partie du stage a été consacrée à des tâches de développement s’inscrivant dans le cadre du projet d'édition de la correspondance entre Jacques Doucet (1853-1929), bibliophile et mécène, figure importante de l'histoire de l'actuel Institut national d'histoire de l'art, et son bibliothécaire René-Jean (1879-1951) sur la période 1908-1929. \\

 \textbf{Abstract :}
    This Master’s thesis was carried out during the second year of the Ecole nationale des chartes’ Master’s in Digital Humanities, entitled “Digital Technologies Applied To History”, concluding a four-month internship at the Institut national d’histoire de l’art (INHA) in Paris, which focused on the technical and managerial aspects of a digital edition project in the field of history of art. This internship was centred around the technical and scientific ramifications and achievements of the super-project PENSE (a “Platform for [hosting] (TEI- and visually) enriched digital editions of [historical] sources”), a project sponsored by both of the departments of the Institute (the research department and the heritage/library department) and aiming to create a single platform giving access to the entirety of the digital edition projects led by INHA research teams. A large part of the internship was devoted to technical tasks related to one specific digital edition project dedicated to the diplomatic and critical edition of the correspondence between Jacques Doucet (1853-1929), fashion designer turned patron and art collector, and a founding figure for the INHA itself and René-Jean (1879-1951) his personal librarian, over a period of 21 years between 1908-1929. \\
	
	\textbf{Mots-clés:} édition numérique~; XML-TEI~; XQuery~; Python~; datavisualisation~; Design Thinking~; Technologies web~; API~; LLM.
	
	\textbf{Informations bibliographiques:} Pauline Charrier, \textit{Entre pratiques éprouvées et ambitions expérimentales, le projet PENSE de l’INHA : une chaîne de traitement des sources de l’histoire de l’art, pour une édition numérique à double vocation, scientifique et patrimoniale}, mémoire de master \enquote{Technologies numériques appliquées à l'histoire}, dir. Jean-Damien Généro, École nationale des chartes, 2024.
	
		\newpage{\pagestyle{empty}\cleardoublepage}