\chapter{Remerciements}
	
\lettrine{M}es remerciements vont tout d’abord à Federico Nurra, chef du Service numérique de la recherche de l’INHA et Jean-Christophe Carius, ingénieur de recherche en charge du projet PENSE et mon superviseur technique pour ce stage, pour leur aide généreuse et leur encadrement attentif, pédagogue et très enrichissant, qui m’ont permis d’acquérir de réelles compétences et de gagner en confiance dans mon usage des outils et des technologies de l’édition numérique.

Je tiens également à remercier Jean-Damien Généro pour ses conseils judicieux quant à l’orientation à donner au plan de ce mémoire ainsi que pour ses recommandations bibliographiques.

Je veux aussi exprimer ma gratitude à l’égard d’Emmanuelle Bermès pour sa patience, ses encouragements et sa compréhension lors de la difficile phase de la rédaction.

Il me semble aussi important de remercier les chercheurs avec lesquels j’ai eu l’occasion de travailler dans le cadre du projet PENSE, dont Marie-Anne Sarda et Victor Claass, pour les échanges riches que nous avons eus et qui m’ont permis de mieux saisir les enjeux du dialogue ingénieur/chercheur dans les projets en humanités numériques.

Enfin, je souhaite remercier les collègues du Service numérique de la recherche pour l’excellente atmosphère de travail ainsi que mes camarades de la promotion 2024 pour leur enthousiasme, leur dynamisme et leur esprit d’entraide au cours de ces deux années de master.

	\newpage{\pagestyle{empty}\cleardoublepage}