\begin{minted}[linenos,frame=lines,fontsize=\small]{python}
import csv
import json
import requests
from collections import defaultdict
import geopy
from geopy.geocoders import Nominatim
from geopy.exc import GeocoderTimedOut


geolocator = Nominatim(user_agent="geojson_generator")

def geocoder_adresse(address):
    try:
        localisation = geolocator.geocode(address)
        if localisation:
            return (localisation.latitude, localisation.longitude)
    except GeocoderTimedOut:
        return geocoder_adresse(address)
    return None

def typer_lieux(types):
    classification = set()
    if 'villegiature' in types or 'domicile' in types:
        classification.add('Residences')
    if 'voyage' in types:
        classification.add('Voyages')
    if 'fournisseurs' in types or 'marchand' in types:
        classification.add('Fournisseurs')
    if 'corresp' in types or 'pm' in types:
        classification.add('Collaborateurs')
    if 'baa' in types:
        classification.add('BAA')
    if 'rj' in types:
        classification.add('ReneJean')
    return list(classification)

# Charger les données CSV depuis l'URL (ça marche avec localhost en brut mais c'est pas opti)
csv_url = 'http://localhost:8080/LettresDoucetReneJean/map/placenames'
response = requests.get(csv_url)
response.raise_for_status()
csv_data = response.text

# "Parser" données CSV
csv_reader = csv.DictReader(csv_data.splitlines(), delimiter=';')
placenames_data = defaultdict(list)
for row in csv_reader:
    placenames_data[row['placeName']].append(row['id'])

csv_file_path = 'lieux_doucet.csv'
geojson_data = {
    "type": "FeatureCollection",
    "features": []
}

# Lecture du fichier CSV local (csv d'origine)
with open(csv_file_path, mode='r', encoding='utf-8') as fichier_csv_local:
    csvreader = csv.DictReader(fichier_csv_local, delimiter=';')
    for row in csvreader:
        nom_lieu = row['nom_lieu']
        nom_type = row['nom_type']
        date_voyage = row['date_voyage']
        lieu = row['lieu']

        types = [t.strip() for t in nom_type.split(',')]
        types_classifies = typer_lieux(types)
        
        address = nom_lieu
        coordinates = geocoder_adresse(address)
        
        if coordinates:
            # Ajouter les ids correspondants aux placeNames au document
            document_ids = '; '.join(placenames_data.get(lieu, [])) #séparer par un point-virgule (car virgules dans les id) et un espace pour permettre le wrap dans l'infobulle
            
            feature = {
                "type": "Feature",
                "geometry": {
                    "type": "Point",
                    "coordinates": [coordinates[1], coordinates[0]]
                },
                "properties": {
                    "nom_lieu": nom_lieu,
                    "nom_type": ', '.join(types_classifies),
                    "date_voyage": date_voyage,
                    "lieu": lieu,
                    "document": document_ids
                }
            }
            geojson_data["features"].append(feature)

# Sauvegarde du GeoJSON dans un fichier
path_fichier_geojson = 'lieux3.geojson'
with open(path_fichier_geojson, 'w', encoding='utf-8') as fichier_geojson:
    json.dump(geojson_data, fichier_geojson, ensure_ascii=False, indent=4)

print(f"Fichier obtenu : {path_fichier_geojson}")
\end{minted}